\documentclass[journal,12pt,twocolumn]{IEEEtran}
%
\usepackage{setspace}
\usepackage{gensymb}
\usepackage{xcolor}
\usepackage{caption}
%\usepackage{subcaption}
%\doublespacing
\singlespacing

%\usepackage{graphicx}
%\usepackage{amssymb}
%\usepackage{relsize}
\usepackage[cmex10]{amsmath}
\usepackage{mathtools}
%\usepackage{amsthm}
%\interdisplaylinepenalty=2500
%\savesymbol{iint}
%\usepackage{txfonts}
%\restoresymbol{TXF}{iint}
%\usepackage{wasysym}
\usepackage{amsthm}
\usepackage{enumerate}
\usepackage{mathrsfs}
\usepackage{txfonts}
\usepackage{stfloats}
\usepackage{cite}
\usepackage{cases}
\usepackage{subfig}
%       \usepackage[latin1]{inputenc}
       \usepackage{fullpage}
       \usepackage{color}
       \usepackage{array}
       \usepackage{calc}
       \usepackage{multirow}
       \usepackage{hhline}
       \usepackage{ifthen}


%\usepackage{xtab}
\usepackage{longtable}
\usepackage{multirow}
%\usepackage{algorithm}
%\usepackage{algpseudocode}
%\usepackage{enumitem}
\usepackage{mathtools}
%\usepackage{iithtlc}
%\usepackage[framemethod=tikz]{mdframed}
\usepackage{listings}
\usepackage{tikz}
\usepackage{hyperref}
%\usepackage{stmaryrd}


%\usepackage{wasysym}
%\newcounter{MYtempeqncnt}
\DeclareMathOperator*{\Res}{Res}
%\renewcommand{\baselinestretch}{2}
\renewcommand\thesection{\arabic{section}}
\renewcommand\thesubsection{\thesection.\arabic{subsection}}
\renewcommand\thesubsubsection{\thesubsection.\arabic{subsubsection}}

\renewcommand\thesectiondis{\arabic{section}}
\renewcommand\thesubsectiondis{\thesectiondis.\arabic{subsection}}
\renewcommand\thesubsubsectiondis{\thesubsectiondis.\arabic{subsubsection}}

% correct bad hyphenation here
\hyphenation{op-tical net-works semi-conduc-tor}
\def\inputGnumericTable{}                                 %%
\lstset{
%language=Python,
frame=single, 
breaklines=true,
columns=fullflexible,
literate = {-}{-}1
}

%\lstset{
	%%basicstyle=\small\ttfamily\bfseries,
	%%numberstyle=\small\ttfamily,
	%language=Octave,
	%backgroundcolor=\color{white},
	%%frame=single,
	%%keywordstyle=\bfseries,
	%%breaklines=true,
	%%showstringspaces=false,
	%%xleftmargin=-10mm,
	%%aboveskip=-1mm,
	%%belowskip=0mm
%}

%\surroundwithmdframed[width=\columnwidth]{lstlisting}


\begin{document}
%

\theoremstyle{definition}
\newtheorem{theorem}{Theorem}[section]
\newtheorem{problem}{Problem}
\newtheorem{proposition}{Proposition}[section]
\newtheorem{lemma}{Lemma}[section]
\newtheorem{corollary}[theorem]{Corollary}
\newtheorem{example}{Example}[section]
\newtheorem{definition}{Definition}[section]
%\newtheorem{algorithm}{Algorithm}[section]
%\newtheorem{cor}{Corollary}
\newcommand{\BEQA}{\begin{eqnarray}}
\newcommand{\EEQA}{\end{eqnarray}}
\newcommand{\define}{\stackrel{\triangle}{=}}
\bibliographystyle{IEEEtran}
%\bibliographystyle{ieeetr}
\providecommand{\nCr}[2]{\,^{#1}C_{#2}} % nCr
\providecommand{\nPr}[2]{\,^{#1}P_{#2}} % nPr
\providecommand{\mbf}{\mathbf}
\providecommand{\pr}[1]{\ensuremath{\Pr\left(#1\right)}}
\providecommand{\qfunc}[1]{\ensuremath{Q\left(#1\right)}}
\providecommand{\sbrak}[1]{\ensuremath{{}\left[#1\right]}}
\providecommand{\lsbrak}[1]{\ensuremath{{}\left[#1\right.}}
\providecommand{\rsbrak}[1]{\ensuremath{{}\left.#1\right]}}
\providecommand{\brak}[1]{\ensuremath{\left(#1\right)}}
\providecommand{\lbrak}[1]{\ensuremath{\left(#1\right.}}
\providecommand{\rbrak}[1]{\ensuremath{\left.#1\right)}}
\providecommand{\cbrak}[1]{\ensuremath{\left\{#1\right\}}}
\providecommand{\lcbrak}[1]{\ensuremath{\left\{#1\right.}}
\providecommand{\rcbrak}[1]{\ensuremath{\left.#1\right\}}}
\theoremstyle{remark}
\newtheorem{rem}{Remark}
\newcommand{\sgn}{\mathop{\mathrm{sgn}}}
\providecommand{\abs}[1]{\left\vert#1\right\vert}
\providecommand{\res}[1]{\Res\displaylimits_{#1}} 
\providecommand{\norm}[1]{\lVert#1\rVert}
\providecommand{\mtx}[1]{\mathbf{#1}}
\providecommand{\mean}[1]{E\left[ #1 \right]}
\providecommand{\fourier}{\overset{\mathcal{F}}{ \rightleftharpoons}}
%\providecommand{\hilbert}{\overset{\mathcal{H}}{ \rightleftharpoons}}
\providecommand{\system}{\overset{\mathcal{H}}{ \longleftrightarrow}}
	%\newcommand{\solution}[2]{\textbf{Solution:}{#1}}
\newcommand{\solution}{\noindent \textbf{Solution: }}
\providecommand{\dec}[2]{\ensuremath{\overset{#1}{\underset{#2}{\gtrless}}}}
%\numberwithin{equation}{subsection}
\numberwithin{equation}{problem}
%\numberwithin{problem}{subsection}
%\numberwithin{definition}{subsection}
%\makeatletter
%\@addtoreset{figure}{problem}
%\makeatother
%
%\let\StandardTheFigure\thefigure
%%\renewcommand{\thefigure}{\theproblem.\arabic{figure}}
%\renewcommand{\thefigure}{\theproblem}
%\numberwithin{figure}{subsection}
\def\putbox#1#2#3{\makebox[0in][l]{\makebox[#1][l]{}\raisebox{\baselineskip}[0in][0in]{\raisebox{#2}[0in][0in]{#3}}}}
     \def\rightbox#1{\makebox[0in][r]{#1}}
     \def\centbox#1{\makebox[0in]{#1}}
     \def\topbox#1{\raisebox{-\baselineskip}[0in][0in]{#1}}
     \def\midbox#1{\raisebox{-0.5\baselineskip}[0in][0in]{#1}}
\vspace{3cm}
\title{
\logo{
LED control using ESP8266
}
}%
\author{Alok Ranjan Kesari and G. V. V. Sharma% 
%\author{Alok Ranjan Kesari$^{1}$ and Dr. G. V. V. Sharma$^{2}$% 
%\thanks{$^{1}$Alok Ranjan Kesari was an intern with the Department of Electrical Engineering, IIT Hyderabad
 %       {\tt\small alok.kesari@yahoo.co.in}}%
%\thanks{$^{2}$Dr. G. V. V. Sharma is with the Department of Electrical Engineering, IIT Hyderabad
%        {\tt\small gadepall@iith.ac.in}}%
\thanks{ The authors are with the Department of Electrical Engineering, IIT Hyderabad
        {\tt\small alok.kesari@yahoo.co.in, gadepall@iith.ac.in}}%
}
% make the title area
%\maketitle
%\newpage
\tableofcontents
 
%%%%%%%%%%%%%%%%%%%%%%%%%%%%%%%%%%%%%%%%%%%%%%%%%%%%%%%%
\begin{abstract}
This manual provides an introduction to Verilog programming using the Icoboard-Lattice FPGA.
\end{abstract}
Download the manual and codes from below link
\begin{lstlisting}
https://github.com/pratibha444/icoboard
\end{lstlisting}
\section{Components}
The necessary components for this manual are listed in Table 
\begin{table}[!h]
\centering
\input{./tabel/components}
\caption{}
\label{table:components}
\end{table}
\section{Software Setup}
\subsection{Icoboard}
For installing icoboard
Open a terminal and execute the following commands.

\begin{lstlisting}
git clone https://github.com/WiringPi/WiringPi.git
cd WiringPi && ./build
sudo apt install build-essential clang bison flex libreadline-dev gawk tcl-dev libffi-dev mercurial graphviz xdot pkg-config libftdi-dev

#Icoprog
#On termuxarch run as root user
#With termuxarch and pizero, this is the only tool required at the pi
svn co http://svn.clifford.at/handicraft/2015/icoprog
cd icoprog && make install

#Icestorm
#On termuxarch run as normal user without sudo
git clone https://github.com/cliffordwolf/icestorm
cd icestorm && make -j4 && sudo make install

#arachne-pnr
#On termuxarch run as normal user without sudo
git clone https://github.com/cseed/arachne-pnr
cd arachne-pnr && make -j4 && sudo make install

#Yosys
#On termuxarch run as normal user without sudo
git clone https://github.com/cliffordwolf/yosys
cd yosys && make -j4 && sudo make install

\end{lstlisting}
\section{Hardware Setup}
\subsection{4 bit binary input}
\begin{itemize}


\item The hardware connections between the Icoboard and Rasberry Pi 4 are available in below figures.\\

\item In figure 2 and 3 the Icoboard and rasberry pi connections are shown. And fig 4 shows the Rasberry Pi pin configuration.

\item Place icoboard on Raspberry Pi 4 and make the connections according to following steps:
\end{itemize}

\begin{itemize}
\item Take the wires and connect them to A5,A2,C3,B4 of the icoboard . These pins are used to give input manually.\\
\item Similarly make connection to GND pin and 3.3V pin of Icoboard\\
\item Connect GND and 3.3V pin on the bread board\\
\item Give the binary input using input pins .\\
\item For example connect all the input pins to GND pin on bread board\\
\item Now open the terminal give the following commands\\
\begin{lstlisting}
cd ICOBOARD
cd codes
cd Binary
make v_fname=binary
python binary.py
\end{lstlisting} 
\item The output is displayed on termianal as 0\\
\item Similarly you can change the values.
\item  If 1 is to be given as input connect it to 3.3V pin and if 0 is to be given as input then connect it to GND pin.
\end{itemize}
\subsection{Seven segment display}
Make the connections according to the Tabel II.\\
The pin configuration of seven segment display is shown in fig 1.
\begin{table}[!h]
\centering
\input{./tabel/connection.tex}
\caption{}
\label{table:components}
\end{table}
\begin{itemize}
\item Open the terminal and execute the following commands
\begin{lstlisting}
cd ICOBOARD
cd codes
cd seven
make v_fname=seven
python seven.py
\end{lstlisting}
\begin{figure}[!h]
\centering
\input{./tabel/seven.tex}
\caption{SSD pin configuration}
\label{fig:arduino}
\end{figure}
\begin{figure}[!h]
\centering
\includegraphics[scale=0.5]{./tabel/icoboard2.eps}
\caption{Icoboard pin configuration}
\label{fig:arduino}
\end{figure}
\begin{figure}[!h]
\centering
\includegraphics[scale=0.5]{./tabel/icoboard1.eps}
\caption{Icoboard pin configuration}
\label{fig:arduino}
\end{figure}
\begin{figure}[!h]
\centering
\includegraphics[scale=0.8]{./tabel/raspberry.eps}
\caption{RasberryPi pin configuration}
\label{fig:arduino}
\end{figure}
\end{itemize}




\end{document}